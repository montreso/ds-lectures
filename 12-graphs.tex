\input templates/header

\usepackage{comment}
\usepackage{listings}


\newcommand{\Mapper}{\textsf{Mapper}\xspace}
\newcommand{\Reducer}{\textsf{Reducer}\xspace}
\newcommand{\Key}{\textit{key}\xspace}
\newcommand{\Values}{\textit{values}\xspace}
\SetKwData{EmitIntermediate}{EmitIntermediate}
\SetKwData{Emit}{Emit}
\SetKw{New}{new}
\SetKwFor{PROCEDURE}{}{}{}

\institute[UniTN]{University of Trento, Italy}
\author{Alessio Guerrieri}
\date[23.03.2016]{March 23, 2016}
\title[DS - Graphs]{\textbf{Distributed Graph Algorithms}}

\graphicspath{figs/12/}


\AtBeginSection[]{
}

\newcommand\Slide[2]{\begin{frame} \frametitle{#1} #2 \end{frame}}
\newcommand\SlideF[2]{\begin{frame} \frametitle{#1} #2 \end{frame}}
\newcommand\Block[2]{\begin{block}{#1} #2 \end{block}}
\newcommand\DueColonne[4]{\begin{columns}\column{#1\textwidth} #3 \column{#2\textwidth} #4 \end{columns}}
\newcommand\TreColonne[6]{\begin{columns}\column{#1\textwidth} #4 \column{#2\textwidth} #5  \column{#3\textwidth} #6 \end{columns}}
\newcommand\Image[1]{\includegraphics[width=\textwidth]{figs/12/#1}}
\newcommand\ImageI[2]{\begin{center}\includegraphics[width=#2\textwidth]{figs/12/#1}\end{center}}
\newcommand\Lista[1]{\begin{itemize} #1 \end{itemize}}
\newcommand\BlockLista[2]{\Block{#1}{\Lista{#2}}}




\begin{document}

\maketitle

\section{Introduction}
\Slide{Who am I?}
{
    \Lista{
        \item My name is Alessio Guerrieri
        \item Postdoctoral researcher with prof. Montresor
        \item Specialization on \textbf{distributed large-scale graph processing}
    }
    \DueColonne{.6}{.4}
    {
        \BlockLista{Graph $G=(V,E)$}
        {
            \item $V$ set of nodes
            \item $E$ set of edges (links, connections) between pair of nodes
        }    
    }{
        \Image{graph}
    }
    
}

\Slide{Today's topic}
{
	\Lista{
	\item Vertex-centric model for graph computation
	\vfill
	\item Possible applications of vertex-centric model
	\vfill
	\item Vertex-centric graph algorithms 
	\vfill
	
	}
}

\subsection{P2P}

\Slide{Peer-to-Peer network}
{
    \DueColonne{.4}{.6}
    {
        \Lista{
            \item Collection of \alert{peer} (nodes)
            \item Have physical connections
            \item Compute an overlay graph
            \item Nodes know existence of neighbors
            \item Can discover other nodes via peer-sampling techniques
        }
    }
    {
        \Image{../10/overlay}
    }
}

\Slide{Computing in Peer-to-Peer systems}
{
    \vfill
    \BlockLista{Peers might want to run protocols on the network:}{
        \item Finding centrality of peer
        \item Test properties of network
        \item Compute distances
        \item ...
    }
    \vfill
    
    Can we simply reuse distributed algorithms for Computer Networks?
    \vfill
}



\subsection{Distance computation}

\Slide{Example: decentralized distance computation}
{
    \Block{Problem}{
        For each peer $x$ in overlay $G$, compute its distance from target peer $Target$.
    }
    
    \BlockLista{Vertex-centric model}{
        \item Initially each peer only knows its direct neighbors in the overlay
        \item Each peer can send messages to each peer of which it knows the existence
        \item Amount of memory used by each peer should be sub-linear
        \item No control on rate of activity of peers and speed of messages (asynchronous execution)
    }
}

\SlideF{Distributed Breadth-First-Search}
{
        \DueColonne{.5}{.5}{
            \begin{Procedure}
            \caption{D-BFS protocol executed by $p$:}
            \UPON{initialization}{
                \eIf{$p = Target$}{
                    $D_p \gets 0$\;
                    \SEND $\langle D_p + 1 \rangle$ \TO neighbors\;
                }{
                    $D_p \gets \infty$
                }
            }
            \UPON{\RECEIVE $\langle d \rangle$}{
                \If{$d < D_p$}{
                    $D_p \gets d$\;
                    \SEND $\langle D_p+1 \rangle$ \TO neighbors\;                
                }
            }
            \end{Procedure}
        }{
            \only<1|handout:1>{
                \Image{distance/start}
            }\only<2|handout:2>{
                \Image{distance/step1}
            }\only<3|handout:3>{
                \Image{distance/step2}
            }\only<4|handout:4>{
                \Image{distance/step3}
            }\only<5|handout:5>{
                \Image{distance/step4}
            }\only<6|handout:6>{
                \Image{distance/step5}
            }\only<7|handout:7>{
                \Image{distance/step6}
            }
        }
}

\SlideF{Distributed Breadth-First-Search}
{
        \DueColonne{.5}{.5}{
            \begin{Procedure}
            \caption{D-BFS protocol executed by $p$:}
            \UPON{initialization}{
                \eIf{$p = Target$}{
                    $D_p \gets 0$\;
                    \SEND $\langle D_p + 1 \rangle$ \TO neighbors\;
                }{
                    $D_p \gets \infty$
                }
            }
            \UPON{\RECEIVE $\langle d \rangle$}{
                \If{$d < D_p$}{
                    $D_p \gets d$\;
                    \SEND $\langle D_p+1 \rangle$ \TO neighbors\;                
                }
            }
            \REPEAT{every $\Delta$ time units}{
                \SEND $\langle D_p+1 \rangle$ \TO neighbors\;                
            }
            \end{Procedure}
        }{

            \Image{distance/step6}
        }
}


\Slide{Notes}{
    \BlockLista{Works if:}{
        \item Nodes execute in any order
        \item Messages arrive in any order
        \item Nodes or edges are added
        \item Messages are lost
    }    
    \BlockLista{Fails if:}{
        \item Nodes fail
        \item Edges are removed
    }
}

\section{Thinking distributively}
\subsection{Big Data}
\Slide{New Graph Applications} {
    Many settings in which there is need to analyze graph:
    \Lista{
        \item Web data
        \item Computer networks
        \item Biological networks
        \item Social networks
        \item ...
    }

    Size can be extremely large
}

\Slide{Approaches}
{
    \TreColonne{.48}{.04}{.48}{
        \BlockLista{Centralized/Sequential}{
            \item Collect data in one place
            \item Load it in memory
            \item Analyze it with traditional techniques
        }
    }{}{
        \BlockLista{Decentralized}{
            \item One machine for each node
            \item Construct overlay network
            \item Run vertex-centric protocol
        }
    }
    \vfill
    \begin{center}
    Both unfeasible! These are better:
    \end{center}
    \vfill

    \TreColonne{.48}{.04}{.48}{
        \Block{Centralized/Parallel}{
            Parallel processes and threads in shared memory
        }
    }{}{
        \Block{Distributed system}{
            Multiple (not N) machines, each hosting part of the graph
        }
    }

}

\Slide{Why distributed?}
{

    \DueColonne{.4}{.6}{
        \BlockLista{Disadvantages}
        {
            \item Slower than parallel systems
            \item Younger field of research
            \item Needs distribution of data
        }
        
        \BlockLista{Advantages}
        {
            \item Fault tolerant
            \item Cheaper
            \item Extendible
        }
    }{
        \Image{datacenter}
    }

}

\Slide{What do we want?}
{
    Nice interface for distributed graph algorithms and fast framework
    to run those algorithms
	\vfill
    \underline{Graph analyst are not distributed systems experts!}
	\vfill
    
    Issues with:
    \Lista{
        \item Data replication
        \item Fault tolerance
        \item Message deliverance
        \item ...
    }
    should be dealt by the system!
}

\Slide{Vertex centric}
{
    Why not reuse vertex-centric interface?

    \TreColonne{.48}{.04}{.48}{    
        \BlockLista{User}
        {
            \item Defines data associated to vertex
            \item Defines type of messages
            \item Defines code executed by single vertex when it receives messages
        }
    }{}{
        \BlockLista{System}
        {
            \item Divides input graph between the available machines 
            \item Each machine \emph{simulates} each vertex independently
            \item Takes care of fault tolerance
        }
    }
}

\Slide{Vertex centric distributed system}
{
	\ImageI{pregel.pdf}{.6}
}

\subsection{Pregel and Vertex-Centric frameworks}


\Slide{Pregel}
{
    \DueColonne{.5}{.5}{
        \Lista{    
            \item    Developed by Google

            \item     First large-scale graph processing system with vertex-centric interface
            
            \item   Created for PageRank
            
            \item    Code automatically deployed on thousands of machines
        }
    }{
        \Image{pregel}
        
        { \footnotesize Programming API}
    }
    
}


\Slide{Pregel-like frameworks}
{
    Many:
    
    \Lista{
        \item Giraph: on top of Hadoop
        \item GraphX: on top of Spark
        \item Gelly: on top of Flink
        \item Graphlab: standalone
    }
    
    All frameworks allow user to run vertex-centric programs on distributed systems

}


\begin{frame}[fragile]
\frametitle{Example (Giraph)}

\lstset{language=java,keepspaces=true}
\footnotesize
\begin{lstlisting}
public void compute(Iterable<IntWritable> messages) {
    int minDist = Integer.MAX_VALUE; 
    for (IntWritable message : messages) {
        minDist = Math.min(minDist, message.get());
    }
    if (minDist < getValue().get()) {
        setValue(new IntWritable(minDist));
        IntWritable distance = new IntWritable(minDist + 1)
        for (Edge<...> edge : getEdges()) {
            sendMessage(edge.getTargetVertexId(),distance);
        }
    }
    voteToHalt();
}
\end{lstlisting}

\end{frame}

\Slide{Common characteristics of Vertex-centric models}
{
	\TreColonne{.48}{.04}{.48}{

		\Block{Independence from problem size}
		{
			Code stays the same regardless of
			\Lista{
				\item size of graph
				\item number of machines used
			}
		}
	
	
		\BlockLista{Embedded fault-tolerance}{
			\item Periodic benchmarking
			\item Machines can "steal nodes" from slower machines
		}
	}{}{
		\BlockLista{Distributed File System}
		{
			\item Usually HDFS
			\item Allow all machines to read and write in common space
		}
		
	}
}


\Slide{Synchronicity of Vertex-centric models}
{
	\TreColonne{.48}{.04}{.48}
	{
		\BlockLista{Synchronous}
		{
			\item Execution in rounds
			\item Each vertex receives in round $i$ all messages sent to it in round $i-1$
			\item Better guarantees $\rightarrow$ easiest model
		}
	}{}{
		\BlockLista{Asynchronous}
		{
			\item No rounds
			\item Messages processed without guarantees
			\item More difficult, more efficient			
		}
	
	}
	\ImageI{giraph}{.6}
}

\subsection{Graph partitioning}


\Slide{Graph partitioning}
{
	First step of analysis
	
	Which machine hosts which subset of nodes/edges?
	\TreColonne{.48}{.04}{.48}
	{
		\Block{Balance}{
			Size of partitions should be balanced
		}
	
	}{}{
		\BlockLista{Quality}{
			\item Messages between nodes hosted in same machine are cheap
			\item Messages between machines are costly
		}
	}
	
	\ImageI{part}{.5}
}

\Slide{Heuristic solutions}
{
	Graph partitioning is NP-Hard
	\vfill
	\TreColonne{.48}{.04}{.48}{
		\Block{Hash-based heuristics}
		{
			Node $n$ ends in machine $H(n) \% K$
		}
	}{}{
		\Block{Iterative algorithms}
		{
			Start from random and incrementally improve
		}
	}
	\vfill
	Heuristics can be devised for specific data-sets	

}

\section{Graph algorithms}

\Slide{}{
	\centering \Huge
	Graph Algorithms

}

\Slide{What algorithms are available}
{
	This area is not completely new: there are distributed algorithms for computer networks
	
	But:
	\Lista{
		\item Computer networks are small
		\item Many interesting graph problems are not interesting on computer networks
	}
	
	Some ideas can be taken from research in PRAM (		Parallel Random Access Machine)
}

\Slide{Problems}
{
	\BlockLista{We'll look at:}{
		\item Triangle counting
		\item Connected components
		\item Strongly connected components
		\item Clustering
	}

	\BlockLista{We won't look at:}{
		\item Pagerank
		\item Single-Value decomposition
	}
	
}

\subsection{Triangle counting}
\Slide{Triangle Counting}
{
	\Block{Definition}{
		In an undirected graph Compute, for each node $n$, how many pairs of nodes $(v,w)$ exists such that $(n,v),(v,w),(w,n)$ are edges.
	}
	\DueColonne{.6}{.4}
	{	
		Application on social networks		

		\vspace{5mm}

		Seems easier than it is...
	}{
		\Image{triangles/ex}
	}

}

\Slide{Triangle counting protocol}
{
		Send neighbor-list to neighbors and see which neighbors we have in common
            \begin{Procedure}
            \caption{Executed by node $p$}
            \UPON{initialization}{
				$T_p \gets 0$\;
                \SEND neighbor-list \TO neighbors\;
            }
            \UPON{\RECEIVE $\langle M \rangle$}{ 
				$Common \gets $ neighbor-list $ \bigcap M$\;
				$T_p \gets T_p + |Common|$\;
            }
            \end{Procedure}	

}

\Slide{Issues}
{
	\BlockLista{Multi-counting}{
		\item Each triangle is counted 6 times!
		\item Should make sure that every triangle is counted only once!
	}
	
	\BlockLista{Hubs}
	{
		\item Hubs are nodes with disproportionally large degree
		\item Exists in almost all real graphs
		\item Will send large neighbor-lists to large number of neighbors
		\item Will receive large quantity of messages
	}

}

\Slide{Triangle counting protocol (2)}
{
		Send list of neighbors with smaller id
            \begin{Procedure}
            \caption{Executed by node $p$}
            \UPON{initialization}{
				$T_p \gets 0$\;
            	$M \gets \{ n: n \in $ neighbor-list$, id_n < id_p \}$\;
                \SEND $M$ \TO neighbors\;
            }
            \UPON{\RECEIVE $\langle M \rangle$}{ 
				$Common \gets $ neighbor-list $ \bigcap M$\;
				$T_p \gets T_p + |Common|$ 
            }
            \end{Procedure}	
	Cut messages by half
}


\Slide{Triangle counting protocol (3)}
{
        \begin{Procedure}
            \caption{Executed by node $p$}
            \UPON{initialization}{
				$T_p \gets 0$\;
            	\ForEach{$n \in$ neighbor-list}
            	{
            	\If{$id_n < id_p$}
            	{
	            	$M \gets \{ q: q \in $ neighbor-list$, id_q < id_n \}$\;
    	            \SEND $LIST(M)$ \TO $n$\;
                }
                }
            }
            \UPON{\RECEIVE $\langle LIST(M) \rangle$ from $n$}{ 
				$Common \gets $ neighbor-list $ \bigcap M$\;
				$T_p \gets T_p + |Common|$ 
				\SEND $TR(|Common|)$ \TO $n$\;
				\SEND $TR(1)$ \TO $v \in Common$\;
            }
			\UPON{\RECEIVE $\langle TR(t) \rangle$}
			{
				$T_p \gets T_p + t$\;
			}
        \end{Procedure}	
}




\subsection{Connectedness}


\Slide{Connected components (problem)}
{
	\Block{Problem definition}
	{
		Two nodes $v,w$ of an undirected graph are in the same weakly connected component (WCC) if exists a path from $v$ to $w$. 
		
		For each node $n$ compute value $C_n$ such that:
		
		\[ \forall v,w \in V, C_v=C_w \iff WCC_v = WCC_w \]
	}
	
}

\Slide{Connected components (solution)}
{
	\DueColonne{.5}{.5}{
            \begin{Procedure}
            \caption{Executed by node $p$:}
            \UPON{initialization}{
				$C_p \gets p.id$\;
                \SEND $\langle C_p \rangle$ \TO neighbors\;
            }
            \UPON{\RECEIVE $\langle c \rangle$}{
                \If{$c < C_p$}{
                    $C_p \gets c$\;
                    \SEND $\langle c \rangle$ \TO neighbors\;                
                }
            }
            \end{Procedure}
    }{
			\only<1|handout:1>{
                \Image{wcc/start}
            }\only<2|handout:2>{
                \Image{wcc/step1}
            }\only<3|handout:3>{
                \Image{wcc/step2}
            }
    }
	
}

\subsection{Strongly Connected Components}

\Slide{Strongly connected components}
{
	\Block{Problem definition}
	{
		Two nodes $v,w$ of an \emph{directed graph} are in the same strongly connected component (SCC) if exist paths from $v$ to $w$ and from $w$ to $v$ (both directions). 
		
	}
	\BlockLista{Centralized solution}
	{
		\item Single depth-first search visit using Tarjan's algorithm
		\item Two depth-first search visits using Kosaraju's algorithm
		\item Distributed solution much more difficult!
	}
	
}

\Slide{Strongly Connected components (solution)}
{
            \begin{Procedure}
            \caption{SCC: Executed by node $p$:}
            \UPON{initialization}{
				$F_p \gets p.id$ \tcp*{Lowest id of nodes that reaches $p$}
				$B_p \gets p.id$ \tcp*{Lowest id of nodes reachable from $p$}
                \SEND $\langle FOR(F_p) \rangle$ \TO out-neighbors\;
                \SEND $\langle BACK(B_p) \rangle$ \TO in-neighbors\;
            }
            \UPON{\RECEIVE $\langle FOR(c) \rangle$}{ 
                \If{$c < F_p$}{
                    $F_p \gets c$
                    \SEND $\langle FOR(c) \rangle$ \TO out-neighbors\;                
                }
            }
            \UPON{\RECEIVE $\langle BACK(c) \rangle$}{
                \If{$c < B_p$}{
                    $B_p \gets c$ 
                    \SEND $\langle BACK(c) \rangle$ \TO in-neighbors\;                
                }
            }
            \end{Procedure}	
}
\Slide{Properties}
{
	\DueColonne{.5}{.5}{
		$B_p$ = minimum id of nodes reachable from $p$

		$F_p$ = minimum id of nodes that can reach $p$		
		\ImageI{scc/start}{.8}
	}{
	
			\only<2|handout:1>{
                \Image{scc/step1}
            }\only<3|handout:2>{
                \Image{scc/step2}
            }
	}
}

\Slide{Properties}
{
	\Block{Lemma1}
	{
		$(F_p = B_p = i) \rightarrow $ p and i are in the same SCC 
	}
	\vfill
	\Block{Lemma2}
	{
		$(F_p = F_q$ and $B_P = B_1 \rightarrow $ p and q are in the same SCC
	}	

	\vfill
	Are the two lemmas correct?
}



\Slide{Properties}
{
	\DueColonne{.5}{.5}{
		$B_p$ = minimum id of nodes reachable from $p$

		$F_p$ = minimum id of nodes that can reach $p$		
		\ImageI{scc/Bstart}{.8}
	}{
	
                \Image{scc/Bstep2}
	}
}


\Slide{Complete algorithm}
{

	\Block{Complete algorithm}
	{
		While graph not empty:
		\begin{itemize}
			\item Run SCC algorithm
			\item remove each node $p$ such that $F_p = B_p$
		\end{itemize}
	}
	\vfill
	Can improve number of SCC found through heuristics
	\vfill
	Quicker convergence for largest SCC
}



\subsection{Graph clustering}

\Slide{Graph clustering}
{
	\DueColonne{.6}{.4}
	{
		Given a graph, divide its vertices in \emph{clusters} such that:
		
		\begin{itemize}
			\item Most edges connect nodes in same cluster
			\item Few edges go across cluster
			\item Clusters are significant
		\end{itemize}
		Precise definition depends from application scenario
		
		All versions are NP-Complete
	}{
		\Image{cluster/clusters}
	}
}

\Slide{Label-propagation algorithm}
{
            \begin{Procedure}
            \caption{Executed by node $p$:}
            \UPON{initialization}{
				$C_p \gets p.id$ \tcp*{starting label equal to id}
				$N_p \gets \{\}$ \tcp*{dictionary containing labels of neighbors}
				\SEND $C_p$ \TO neighbors\;
            }
            \UPON{\RECEIVE c \FROM q}{
            	$N_p[q] = c$ \tcp*{update dictionary with new label of neighbor}
            }
			\REPEAT{every $\Delta$ time units}{
				$c' \gets$ most common label in neighbors\;
				\If{$C_p \neq c'$}{
					\SEND $c'$ \TO neighbors\;
					$C_p \gets c'$\; 
				}
			}
            \end{Procedure}	
			Note:ties resolved randomly
}

\Slide{Sample run}
{

	\only<1|handout:1>{
		Initialization:
		
		\Image{cluster/start}
	}
	\only<2|handout:2>{
		Active node chooses label 1 (randomly)
		
		\Image{cluster/step1}
	}
	\only<3|handout:3>{
		Active node chooses label 3(randomly)
		
		\Image{cluster/step2}
	}

	\only<4|handout:4>{
		Active node chooses label 6(randomly)
		
		\Image{cluster/step3}
	}

	\only<5|handout:5>{
		Active node chooses label 6
		
		\Image{cluster/step4}
	}


	\only<6|handout:6>{
		Active node chooses label 1
		
		\Image{cluster/step5}
	}


	\only<7|handout:7>{
		Active node chooses label 1
		
		\Image{cluster/step6}
	}


	\only<8|handout:8>{
		Active node chooses label 6
		
		\Image{cluster/step7}
	}


	\only<9|handout:9>{
		Converged state: no node change its mind		
		\Image{cluster/step8}
	}



}



\Slide{Disadvantages}
{
	\Block{Non-deterministic}
	{
		Non-null probability of collapsing to single cluster
	}
	\Block{Low quality}
	{
		In practice, it's not that good
	}
	\BlockLista{Better options}
	{
		\item Slow, iterative algorithms
		\item Fast, heuristics
		\item My algorithm!
	}
}


\Slide{Conclusions}
{
	\Lista{
		\item Vertex-centric algorithms can be used in:
		\Lista{
			\item P2P systems
			\item Computer networks
			\item Large-scale graph analysis
		}
		\vfill
		\item Field still in flux
		\Lista{
			\item New frameworks
			\item New programming models
			\item New algorithms
		}
		\vfill
		\item Many applications
		\Lista{
			\item Everything is a graph
			\item Could be interesting for a project
		}
	}
}

\Slide{References}
{
    \emph{Pregel: A System for Large-Scale Graph Processing} by Malewicz et al.

    \vfill

    \emph{Pregel Algorithms for Graph Connectivity Problems with
Performance Guarantees} by Yan et al.
	\vfill
	
	\emph{Giraph} \url{http://giraph.apache.org/}
	
	\vfill
	
	\emph{Graphx} \url{http://spark.apache.org/graphx/}
}

\end{document}
